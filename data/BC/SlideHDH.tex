\documentclass{beamer}
% Copyright 2015 by Do Phan Thuan

% Loại mẫu slice
%\usetheme{AnnArbor}
%\usetheme{Antibes}
\usetheme{Boadilla}
%\usetheme{CambridgeUS}
%\usetheme{Hannover}

% Ký tự tiếng Việt
\usepackage[utf8]{vietnam}
% Chèn ảnh
\usepackage{graphicx}
% Chèn đường dẫn 
\usepackage{url}

% Vẽ đồ thị

% Insert code
\usepackage{listings}
\lstset{language=C++,
   %keywords={break,case,catch,continue,else,elseif,end,for,function,
   %   global,if,otherwise,persistent,return,switch,try,while},
   basicstyle=\ttfamily,
   keywordstyle=\color{blue},
   commentstyle=\color{red},
   stringstyle=\color{dkgreen},
   frame=lrtb,
   %frame=5 pt,
   numbers=left,
   numberstyle=\tiny\color{gray},
   stepnumber=1,
   numbersep=10pt,
   backgroundcolor=\color{white},
   tabsize=4,
   showspaces=false,
   showstringspaces=false}
% Tô mầu cho bảng
\usepackage{colortbl}


\usepackage{color}

\definecolor{dkgreen}{rgb}{0,0.6,0}
\definecolor{gray}{rgb}{0.5,0.5,0.5}
\definecolor{mauve}{rgb}{0.58,0,0.82}
  
\definecolor{Xanh}{rgb}{0,0.5,1}
\definecolor{Do}{rgb}{1,0.25,0}
\definecolor{Vang}{rgb}{1,1,0}
\definecolor{Datroi}{rgb}{0,0,1}
% Vẽ hình
\usepackage{tikz}
\usetikzlibrary{arrows,shapes}
% Vẽ mạch điện
\usepackage[siunitx,european resistors]{circuitikz}

% multirow
\usepackage{multirow}

\usepackage{pbox}

% Tô mầu cho bảng
\usepackage{colortbl}
\definecolor{Xanh}{rgb}{0,0.5,1}
\definecolor{Do}{rgb}{1,0.25,0}
\definecolor{Vang}{rgb}{1,1,0}
\definecolor{Datroi}{rgb}{0,0,1}

% Một vài ký hiệu thường dùng
\def\R{{\mathbb R}}
\def\N{{\mathbb N}}
\def\X{{\mathcal X}}
\def\Y{{\mathcal Y}}
\def\F{{\mathcal F}}
\def\P{{\mathcal P}}
\def\E{{\mathbb E}}
\def\I{{\mathbb I}}
\def\sign{{\rm sign}}

% Xác định khoảng dãn trong bảng
%\renewcommand\arraystretch{1.6}

% a few macros
\newcommand{\bi}{\begin{itemize}}
\newcommand{\ei}{\end{itemize}}
\newcommand{\ig}{\includegraphics}
\newcommand{\subt}[1]{{\footnotesize \color{subtitle} {#1}}}

% named colors
\definecolor{offwhite}{RGB}{249,242,215}
\definecolor{foreground}{RGB}{255,255,255}
\definecolor{background}{RGB}{24,24,24}
\definecolor{title}{RGB}{107,174,214}
\definecolor{gray}{RGB}{155,155,155}
\definecolor{subtitle}{RGB}{102,255,204}
\definecolor{hilight}{RGB}{22,155,104}
\definecolor{vhilight}{RGB}{255,111,207}
\definecolor{lolight}{RGB}{155,155,155}
%\definecolor{green}{RGB}{125,250,125}

% Minted
%\usepackage{minted}
%\usemintedstyle{monokai}
%\newminted{cpp}{fontsize=\footnotesize}

% Graph styles
\tikzstyle{vertex}=[circle,fill=black!50,minimum size=15pt,inner sep=0pt, font=\small]
\tikzstyle{selected vertex} = [vertex, fill=red!24]
\tikzstyle{edge} = [draw,thick,-]
\tikzstyle{dedge} = [draw,thick,->]
\tikzstyle{weight} = [font=\scriptsize,pos=0.5]
\tikzstyle{selected edge} = [draw,line width=2pt,-,red!50]
\tikzstyle{ignored edge} = [draw,line width=5pt,-,black!20]

%gets rid of bottom navigation bars
\setbeamertemplate{footline}[frame number]{}

%gets rid of bottom navigation symbols
%\setbeamertemplate{navigation symbols}{}

%gets rid of footer
%will override 'frame number' instruction above
%comment out to revert to previous/default definitions
%\setbeamertemplate{footline}{}

% Tác giả, Tiêu đề, vân vân
\title[]{{\huge \bf Thuật toán lập lịch cho hệ thống thời gian thực} \\}
\author[Đặng Quang Trung]{
Đặng Quang Trung\\
}

\institute[]{
%\inst{1}%
Hệ Điều Hành
}

\logo{\includegraphics[scale=0.05]{hust.jpg} \vspace{220pt}}

\begin{document}

\begin{frame}
\titlepage
\end{frame}
\begin{frame}{Nội dung}
\tableofcontents
\end{frame}
\section{Hệ thống thời gian thực}
\begin{frame}{Hệ thống thời gian thực}
\begin{itemize}
\item[•] Trong thế giới vật lý, mục đích của một hệ thống thời gian thực là có một thực hiện vật lý trong một khung thời gian đã chọn.
\item[•] Thông thường, một hệ thống thời gian thực bao gồm một hệ thống điều khiển( máy tính ) và một hệ thống bị điều khiển ( môi trường ).
\item[•] Mỗi công việc xảy ra trong một hệ thống thời gian thực có một số thuộc tính thời gian. Các thuộc tính thời gian này cần được xem xét khi lập kế hoạch các nhiệm vụ trên một hệ thống thời gian thực.
\begin{itemize}
\item[] Release time (or ready time), Deadline, Minimum delay, Maximum delay, Worst case execution time, Run time, Weight (or priority).
\end{itemize}
\end{itemize}
\end{frame}
\begin{frame}{Định nghĩa}
\begin{itemize}
\item[•] Một hệ thống chứa một tập các tác vụ:
\begin{displaymath}
T = \{\tau_1, \tau_2, \ldots, \tau_n\}
\end{displaymath}
\item[•] Thời gian thực hiện của mỗi tác vụ là $C_i$ với $\tau_i \in T$
\item[•]  Hệ thống được cho là thời gia thực nếu có tồn tại ít nhất tác vụ $\tau_i \in T$, tác vụ rơi vào tình trạng:
\end{itemize}
\end{frame}
\begin{frame}{Tình trạng tác vụ}
\begin{itemize}
\item[1, ] Tác vụ $\tau_i$ là tác vụ hard real-time. Thời gian thực hiện tác vụ $\tau_i$ phải được hoàn thành bởi thời gian hết hạn $D_i ( C_i \leq D_i )$.
\item[2, ] Tác vụ $\tau_i$ là tác vụ \textbf{soft real-time}. Tác vụ cuối cùng $\tau_i$ kết thúc tính toán của nó sau gian gian hết hạn $D_i$, sẽ bị phạt nặng hơn. Hàm phạt $P(\tau_i)$ được định nghĩa cho tác vụ. Nếu $C_i \leq D_i$ hàm phạt $P(\tau_i)$ bằng 0 trái lại $P(\tau_i) > 0$ giá trị này tăng theo $C_i - D_i$.
\item[3, ] Tác vụ là \textbf{firm real-time}. Tác vụ kết thúc công việc tính toán của nó sớm hơn thời hạn hết hạn $D_i$, sẽ nhận được thưởng. Hàm thưởng $R(\tau_i)$ được định nghĩa cho tác vụ. Nếu $C_i \geq D_i$, hàm $R(\tau_i)$ bằng 0, trái lại $R(\tau_i) > 0$ giá trị này tăng theo $D_i - C_i$.
\end{itemize}
\end{frame}
\begin{frame}{Hàm thưởng và phạt của hệ thống}
\begin{itemize}
\item[•] Một tập các tác vụ thời gian thực $T = \{\tau_1, \tau_2, \ldots , \tau_3\}$ có thể là hỗn hợp của các tác vụ hard, soft, firm thời gian thực.
\item[•] $T_s$ là tập tất cả các tác vụ \textbf{soft real-time} trong T, ví dụ $T_s = \{\tau_{s,1}, \tau_{s,2}, \ldots , \tau_{s,l}\}$ với $\tau_{s,i} \in T$. Hàm lỗi của hệ thống sẽ là $P(T)$
\begin{displaymath}
P(T) = \sum_{i-1}^{l} P(\tau_{s,i})
\end{displaymath}
\item[•] $T_f$ là tập tất cả các tác vụ \textbf{firm real-time} trong T, ví dụ $T_f = \{\tau_{f,1}, \tau_{f,2}, \ldots , \tau_{f,k}\}$ với $\tau_{f,i} \in T$. Hàm thưởng của hệ thống sẽ là $R(T)$
\begin{displaymath}
R(T) = \sum_{i-1}^{k} R(\tau_{f,i})
\end{displaymath}
\end{itemize}
\end{frame}
\begin{frame}{Giới thiệu}
\begin{itemize}
\item[•] Mục địch của lập lịch
\begin{itemize}
\item Tối ưu hóa hiệu năng của hệ thống, thông lượng.
\item Làm thế nào? sắp xếp các tiến trình.
\item Thuật toán lập lịch Cố gắng để đạt được kết quả tốt nhất có thể (trung bình hoặc cho một bộ quy trình nhất định)
\end{itemize}
\item[•] Tiêu chí tối ưu
\begin{itemize}
\item Thời gian đáp ứng, thời gian chờ đợi, thời gian đáp ứng của quá trình.
\item Sử dụng bộ vi xử lý.
\item Thông lượng (quy trình hoàn thành cho mỗi đơn vị thời gian).
\item Đáp ứng các thời hạn: tính khả thi, khả năng lập lịch
\end{itemize}
\end{itemize}
\end{frame}
\section{Phân lớp thuật toán}
\begin{frame}{Phân lớp thuật toán lập lịch}
\begin{itemize}
\item[•]Static
\bi
\item Tất cả các thông số về độ ưu tiên được biết trước.
\item Lập lịch có thể được hoàn thành trong thời gian thiết kế, do đó hiệu quả cao trong thời gian chạy.
\item Áp dụng cho các hệ thống nhúng đóng.
\item 
\ei
\item[•] Dynamic
\bi
\item 
\item Không phải tất cả các tham số đều được biết trước
\item Lập lịch trực tuyến.
\item Mục tiêu tối ưu thường có thể đạt được chỉ khoảng.
\item Lập lịch yêu cầu hiệu năng tính toán.
\item Hoạt động hệ điều hành thông thường cho hệ thống tương tác
\ei 
\end{itemize}
\end{frame}
\begin{frame}{Phân lớp thuật toán lập lịch (2)}
\begin{itemize}
\item Lập lịch không trưng dụng.
\bi
\item Hệ điều hành không thu hồi bộ xử lý từ tiến trình đang chạy, nó chạy cho đến khi nó kết thúc hoặc trả lại bộ xử lý.
\item Ưu điểm: đơn giản thực hiện thay đổi quy trình.
\item Nhươc điểm: không hữu ích cho các hệ thống thời gian thực, nơi mà phản ứng nhanh với các sự kiện bên ngoài là cần thiết
\ei
\item Lập lịch trưng dụng
\bi
\item Hệ điều hành sẽ thu hồi bộ xử lý của tiến trình đang chạy bất cứ khi nào có nhu cầu lập lịch lại.
\ei
\end{itemize}
\end{frame}
\begin{frame}{Sở đồ cấu trúc phân lớp tổng quát}
\begin{figure}[h]
\begin{center}
\includegraphics[scale=0.45]{hdh.png}
\caption{Sơ đồ lập lịch}
\end{center}
\end{figure}
\end{frame}
\begin{frame}{Lập lịch hệ thống thời gian thực}
\begin{itemize}
\item[•] Earliest Deadline First(EDF)
\item[•] Rate-monnotonic (RM)
\end{itemize}
\end{frame}
\begin{frame}{Earliest Deadline First(EDF)}
\begin{itemize}
\item[•] Lập lịch dựa trên thời gian hết hạn
\bi
\item Chỉ sử dụng thời gian hết hạn của tiến trình lên kế hoạch, không dungf thời gian tính toán.
\item Bộ xử lý được gán cho tác vụ với thời gian hết hạn gần nhất(thời gian hết hạn sớm nhất, EDF).
\item Không trưng dụng: Lập lịch lại khi tác vụ hiện tại thực hiện xong.
\item Trưng dụng: Lập lịch lại khi có 1 tiến trình mới sẵn sàng.
\ei
\end{itemize}
\end{frame}
\section{Thuật toán EDF}
\begin{frame}{Earliest Deadline First}
\begin{itemize}
\item Static EDF cùng thời gian sẵn sàng.
\bi
\item Sắp xếp các tiến trình với thời gian hết hạn tăng dần.
\item Bất cứ khi nào một tiến trình được lập lịch, kiểm tra xem thời hạn của nó có được đáp ứng hay bị vi phạm.
\item Cho k tiến trình nó thỏa mãn.
\ei
\end{itemize}
\begin{displaymath}
\sum_{i=1}^k C_i \leq D_k  \ \ k = 1 \ldots n
\end{displaymath}
\end{frame}
\begin{frame}{Earliest Deadline First}
\begin{itemize}
\item Ví dụ:
\begin{figure}[h]
\begin{center}
\includegraphics[scale=0.6]{sl.png}
\end{center}
\end{figure}
\item EDF là tối ưu thời gian sẵn sàng giống nhau.
\end{itemize}
\end{frame}
\begin{frame}{Earliest Deadline First}
\bi
\item Chứng minh
\bi
\item Bộ xử lý luôn được sử dụng vì $R_i = 0$.
\item Nếu lập lịch đã được tiến hành đúng thời điểm trước khi thêm $P_k$, sau đó sẽ không thêm được bất kì gì nữa:
\begin{displaymath}
\sum_{i=1}^{k} C_i > D_k
\end{displaymath}
\item Khả năng duy nhất là sau đó trao đổi $P_k$ với một trong các tiến trình đã được lập lịch $P_j$ (j < k).
\item Tổng thời gian thực hiện vẫn không thay đổi.
\item  Chúng ta biết rằng $D_j \leq D_k$ và do đó:
\begin{displaymath}
\sum_{i=1}^{k} C_i > D_j
\end{displaymath}
\item Do đó sự trao đổi làm cho mọi thứ tồi tệ hơn.
\item Nếu thuật toán không tìm được giải pháp nào khác.
\ei
\ei
\end{frame}
\begin{frame}{Earliest Deadline First}
\begin{itemize}
\item Static EDF với thời gian sẵn sàng khác nhau
\bi
\item Không trưng dụng là không tối ưu
\begin{figure}[h]
\begin{center}
\includegraphics[scale=0.6]{sl1.png}
\end{center}
\end{figure}
\ei
\end{itemize}
\end{frame}
\begin{frame}{Earliest Deadline First}
\begin{itemize}
\item Có lập lich tối ưu khác ví du:
\begin{figure}[h]
\begin{center}
\includegraphics[scale=0.6]{sl2.png}
\end{center}
\end{figure}
\item Thuật toán EDF sẽ tối ưu cho trường hợp này nếu có trưng dụng
\item Ví dụ:
\begin{figure}[h]
\begin{center}
\includegraphics[scale=0.6]{sl3.png}
\end{center}
\end{figure}
\end{itemize}
\end{frame}
\section{RM}
% TODO: Book
\begin{frame}
\begin{figure}[h]
\begin{center}
\includegraphics[scale=0.5]{thanks.jpg}
\end{center}
\end{figure}
\end{frame}
\begin{frame}{Tài liệu tham khảo}
\section*{Tài liệu tham khảo}
\end{frame}
\end{document}